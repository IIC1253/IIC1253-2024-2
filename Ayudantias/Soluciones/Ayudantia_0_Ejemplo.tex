% Plantilla para documentos LaTeX para enunciados
% Por Pedro Pablo Aste Kompen - ppaste@uc.cl
% Licencia Creative Commons BY-NC-SA 3.0
% http://creativecommons.org/licenses/by-nc-sa/3.0/

\documentclass[12pt]{article}

% Margen de 1 pulgada por lado
\usepackage{fullpage}
% Incluye gráficas
\usepackage{graphicx}
% Packages para matemáticas, por la American Mathematical Society
\usepackage{amssymb}
\usepackage{amsmath}
% Desactivar hyphenation
\usepackage[none]{hyphenat}
% Saltar entre párrafos - sin sangrías
\usepackage{parskip}
% Español y UTF-8
\usepackage[spanish]{babel}
\usepackage[utf8]{inputenc}
% Links en el documento
\usepackage{hyperref}
\usepackage{fancyhdr}
\setlength{\headheight}{15.2pt}
\setlength{\headsep}{5pt}
\pagestyle{fancy}

\newcommand{\N}{\mathbb{N}}
\newcommand{\Exp}[1]{\mathcal{E}_{#1}}
\newcommand{\List}[1]{\mathcal{L}_{#1}}
\newcommand{\EN}{\Exp{\N}}
\newcommand{\LN}{\List{\N}}

\newcommand{\comment}[1]{}
\newcommand{\lb}{\\~\\}
\newcommand{\eop}{_{\square}}
\newcommand{\hsig}{\hat{\sigma}}
\newcommand{\ra}{\rightarrow}
\newcommand{\lra}{\leftrightarrow}

% Cambiar por nombre completo + número de alumno
\newcommand{\alumno}{Nombre Apellido - 10000001}
\rhead{Tarea Ejemplo - \alumno}

\begin{document}
\thispagestyle{empty}
% Membrete
% PUC-ING-DCC-IIC1103
\begin{minipage}{2.3cm}
\includegraphics[width=2cm]{img/logo.pdf}
\vspace{0.5cm} % Altura de la corona del logo, así el texto queda alineado verticalmente con el círculo del logo.
\end{minipage}
\begin{minipage}{\linewidth}
\textsc{\raggedright \footnotesize
Pontificia Universidad Católica de Chile \\
Departamento de Ciencia de la Computación \\
IIC1253 - Matemáticas Discretas \\}
\end{minipage}


% Titulo
\begin{center}
\vspace{0.5cm}
{\huge\bf Tarea Ejemplo}\\
\vspace{0.2cm}
\today\\
\vspace{0.2cm}
\footnotesize{2º semestre 2024 - Profesores P. Bahamondes - D. Bustamante - M. Romero}\\
\vspace{0.2cm}
\footnotesize{\alumno}
\rule{\textwidth}{0.05mm}
\end{center}



\section*{Respuestas}
% Estas numeracion es solo de ejemplo

\subsection*{Pregunta 1}
\subsubsection*{Pregunta 1.1}
% Respuesta pregunta
Se demostrará por inducción simple.

\textbf{BI:} Para $n = 0$ se tiene que:
\begin{itemize}
    \item LI: $\sum\limits_{i=0}^n i^2 = \sum\limits_{i=0}^0 i^2 = 0$
    \begin{equation*}
        \sum_{i=0}^n i^2 = \sum_{i=0}^0 i^2 = 0
    \end{equation*}
    \item LD: $\frac{n(n + 1)(2n + 1)}{6} = \frac{0(1)(1)}{6} = 0$
    
    LI = LD por lo que el caso base se cumple.
\end{itemize}

\textbf{HI:} Supongamos que la propiedad se cumple para $n$, es decir, que
\begin{equation*}
    \sum_{i=0}^n i^2 = \frac{n(n + 1)(2n + 1)}{6}
\end{equation*}

\textbf{TI:} PD: La propiedad se cumple para $n + 1$, es decir, que 
\begin{equation*}
    \sum_{i=0}^{n + 1} i^2 = \frac{(n + 1)((n + 1) + 1)(2(n + 1)+ 1)}{6}
\end{equation*}

Por HI, se tiene que
\begin{align*}
    \sum_{i=0}^n i^2 &= \frac{n(n + 1)(2n + 1)}{6} \\
    \sum_{i=0}^n i^2 + (n + 1)^2 &= \frac{n(n + 1)(2n + 1)}{6} + (n + 1)^2 \\
    \sum_{i=0}^{n + 1} i^2 &= \frac{n(2n^2 + n + 2n + 1) + 6(n + 1)^2}{6} \\
    \sum_{i=0}^{n + 1} i^2 &= \frac{2n^3 + 3n^2 + n + 6n^2 + 12n + 6}{6} \\
    \sum_{i=0}^{n + 1} i^2 &= \frac{2n^3 + 9n^2 + 13n + 6}{6} \\
    \sum_{i=0}^{n + 1} i^2 &= \frac{(n + 1)(2n^2 + 7n + 6)}{6} \\
    \sum_{i=0}^{n + 1} i^2 &= \frac{(n + 1)(2n + 3)(n + 2)}{6} \\
    \sum_{i=0}^{n + 1} i^2 &= \frac{(n + 1)((n + 1) + 1)(2(n + 1) + 1)}{6} \\
\end{align*}
que es lo que queríamos demostrar.

Concluímos que $\sum\limits_{i = 0}^n i^2 = \frac{n(n + 1)(2n + 1)}{6}$.$_\square$


% Fin del documento
\end{document}

